\documentclass[paper=a4, twoside=false, fontsize=12pt, parskip=half,
                bibliography=totoc, listof=totoc]{scrbook}

% General packages
\usepackage{sourdough}

% Basic attributes
\author{Hendrik Kleinwächter}
\title{The Sourdough Framework}

\begin{document}
%\input{cover/cover.tex}
\titlepage

{%
 \hypersetup{hidelinks}
 \ifdefined\HCode\else\tableofcontents\fi
}

\chapter{ebook UTF-8}
Dès Noël, où un zéphyr haï me vêt de glaçons würmiens, je dîne d’exquis rôtis
de bœuf au kir, à l’aÿ d’âge mûr, \&cætera.  Jörg bäckt quasi zwei Haxenfüße
vom Wildpony.

\chapter{ebook ASCII}
If you're a hobby brewer, you'll know that it's important to keep your beer at
certain temperatures to allow the different amylases to convert the contained
starches into sugar~\cite{beer+amylase}.
This test, called the \emph{Iodine Starch Test}, involves mixing iodine into
a sample of your brew and checking the color.

\chapter{ebook Pictures}
\begin{figure}[!htb]
  \includegraphics[width=\textwidth]{baking-experiment-temperatures.png}
  \caption[Surface temperature for different steaming methods]{png file}
\end{figure}

\printbibliography

\end{document}
